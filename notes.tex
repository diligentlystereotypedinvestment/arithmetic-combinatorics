\documentclass[a4paper]{article}

\input{~/templates/math.tex}
\newtheorem{question}{Question}
\usepackage{fullpage}

\title{Arithmetic Combinatorics}
\author{Vincent Tran}

\begin{document}
\maketitle

\section{3/19 - Elementary Methods}

\subsection{Inverse Theorems}

We will look at sum sets, product sets, and a few times quotient sets.

The context for this will be $G $, an abelian group. We are interested in $\Z, \Z^d, \R^d, \Z_p \coloneqq \Z / p\Z, \Z_2^n$. Let $A,B,C $ be finite subsets of $G $. In addition, all sets will be non-empty.

\begin{definition}
	\textbf{Minkowski Sum}
	\[
		A + B = \{a+b|a\in A, b\in B\}
	.\]
	\[
		A - B = \{a-b|a\in A, b\in B\}
	.\]

	Clearly $A+B $ is associative and commutative.

	Similarly,
	\[
		nA \coloneqq A + \cdots + A
	\]
	$n $ times.
\end{definition}

\begin{property}
	$nA - mA \ne (n-m)A $ in general, i.e. the Minkowski sum doesn't distribute.
\end{property}

\begin{question}
	When is $A+A $ small?
\end{question}

Trivially, we have
\[
	|A| \le |A+A| \le |A|^2
.\]

We are looking for when $|A+A| $ is close to $|A| $.

Direct Problem: How small can $|A+A| $ be?

Inverse: For which $A $ is $|A+A| $ small?

\begin{example}
	Let $G = \Z $, $A = [0,n-1] $.
	Then $A+A = [0,2n-2] \implies |A+A| = 2|A|-1 $.

	By using affine transformations $\phi: \Z\to \Z $ defined by $\phi(x) = ax+b $, we have that $b, b+r, \ldots, b+(n-1)r $ is a value for $|A| $ s.t. $|A+A| = 2|A| -1 $.
\end{example}

\begin{definition}
	\textbf{Generalized Arithmetic Progressions}: They are of the form $b + r_{1}x_{1}+\cdots + r_dx_d $ where $x_i \in [0,n_i-1]$, i.e. affinely transformed sums of $A_i $.
\end{definition}

\begin{thm}
	$\forall A \subseteq \Z |A+A| \ge 2|A|-1$.
\end{thm}

\begin{proof}
	Proof 1: Induction. Let $A = \{a_{1} < a_{2} < \cdots < a_n\}   $ and define $A' = A \setminus \{a_n\}   $.
	By the induction hypothesis, $|A'+A'| \ge 2|A'| - 1 = 2n-3 $.
	By noticing that $2a_n, a_n + a_{n-1} $ are elements not in $A'+A' $ but are in $A+A $, we have that $|A+A| \ge 2|A| - 1 $.

	Proof 2: Go bi. We can generalize this theorem to more variables: $|A+B| \ge |A| + |B| - 1 $, in which case induction is ever simpler as only the largest element is needed.

	Proof 3: Take two sets $A,B $, represent them as cicles with elements in descending order in them.
	Take $|B| $ picks in $A $ and $|A| $ picks in $B $.
	Draw a bar connecting a pick in $A $ to a pick in $B $.
	Do this for all picks in $B $.
	Then do this for all picks in $A $ (above two steps).
	We then have found $|A| + |B| - 1 $ distinct elements (?).
\end{proof}

\begin{thm}[Cauchy-Davenport]
	For $G = \Z_p,\ |A+B| \ge \min(|A|+|B| - 1, p) $.
\end{thm}

\begin{definition}[\textbf{e-transform}]
	Fix an element $e \in A-B $.
	Then
	\begin{align*}
		A_{(e)} :\coloneqq  A \cup (B+e) \\
		B_{(e)} :\coloneqq B \cap (A - e)
	.\end{align*}
	See \Cref{fig:e-transform}
\end{definition}

\begin{lem}
	\begin{enumerate}
		\item $|A_{(e)}| + |B_{(e)}| = |A| + |B| $
		\item $|A_{(e)}\ge |A| $
		\item $A_{(e)}+B_{(e)} \subseteq A+B $
	\end{enumerate}
\end{lem}
\begin{proof}
	For a), clearly $A_{(e)} $ contains $|A| $ elements plus some.
	Then for any element $b \in B $, $b+e \in A $ or $\not\in A $.

	If $b+e \in A $, then $b \in A-e \implies b \in B_{(e)} $.

	If $b+e \not\in A $, then obviously $b+e \in A_{(e)} $ and not in $A $.
	Thus $|A_{(e)}| + |B_{(e)}| = |A| + |B| $.

	b) is truly trivial.

	c) If we take $a \in A_{(e)} $ and it falls in $A $, then we are done as $B_{(e)} \subseteq B $.
	So the only hard case is when $a \in B+e \setminus A $.
	By definition, $A_{(e)}+B_{(e)} $ consists of $a + b, b \in B_{(e)} $ and hence $b \in A \setminus e $.
	Thus $b = a' - e \implies a + b = a + a'-e = b' + a'$ for $b' \in B $ since $a \in B+e $.
\end{proof}

\begin{proof}[Proof: Cauchy-Davenport]
	If $|B| = 1 $ then we are trivially done.
	We then use induction on the size of $B $ and the e-transform to see that $B_{(e)} = B \forall e \in A \setminus B $.
 	Hence $\forall e \in A- B,\ B+e \subseteq A \implies B + A-B \subseteq A $, i.e. $A + (B-B) \subseteq A $.

	As $|B| \ge 2 $, we can let $d = b_{1}-b_{2} $ and see that $A + d = A + (B-B) = A $ and the same is true for all multiples of $d $ (equality is true since $|A| \le |A+(B-B)| $).
	Since $A = \Z_p $ and we are done (this is the min).
\end{proof}

\begin{thm}[Vosper]
	For $A,B \subseteq \Z $ and $|A+B| = |A| + |B| - 1 $, then $A,B $ are a.p. (arithmetic progressions) with the same step.
\end{thm}
\begin{proof}
	(Proof from Tao, not class)

	First we handle three cases:

	$A $ or $B $ are arithmetic progressions: WLOG say $A $ is.
	Then $A = \{a, a + v, \ldots , a + nv\}   $.
	So then $|B| + n = |A| + |B| - 1 = |A+B|$ by hypothesis.

	WLOG let $A = \{a, a+v, \ldots , a + (n-1)v\}  + \{0,v\}   $ with $v $ positive.
	So $|A+B| = |\{a, \ldots, a + (n-1)v\}  + B + \{0,v\} | \ge n-1 + |B + \{0,v\} |$ by Cauchy-Davenport.
	By Cauchy-Davenport again, we have that $|B+\{0,v\} | \ge |B| + 1  $ and from above ($|B| + n \ge n-1 + |B+\{0,v\} |  $) we have that $|B| + 1 = |B + \{0,v\} |  $.

	The largest element of $B $ (say $b_n $) plus $v $ isn't in $B $, so this is the only element of $B + \{0,v\}$ that isn't in $B $, giving us that $B \setminus \{b_n\} + v \subseteq B$.
	Hence $B $ is an arithmetic progression.

	If $|A+B| $ is an arithmetic progression, then let $C = -(\Z_p \setminus (A+B)) $.
	Notice that $|C| = p - |A+B| = p + 1 - |A| - |B|$.
	It follows that $C $ is an arithmetic progression with the same step because the step is an additive generator in $\Z_p $.
	As such if we continue out the progression and reversed it (by negating), we would get the later half that isn't in $-(A+B) $.

	Next we can see that $C+B \subseteq (\Z_p \setminus A)$ because if $C+B $ intersected $-A $ at say $-a = c+b $, then $-(a + b)$ would be in $-(A+B) $ but also $C $, a contradiction.
	Hence $p - |A| \ge |C+B| \ge |C|+|B| - 1 = p - |A|$ by Cauchy-Davenport.
	So $|C+B| = p - |A| = |C| + |B| - 1$.
	By the work before, this gives us that $B $ is an arithmetic progression of the same step as $C $.
	Similarly for $A $.

	Now to prove this for when none of them are arithmetic progressions.
	We use induction.
	For $|B| = 2 $, we have that $B $ is an arithmetic progression and we are done.

	$|B| > 2 $:
	We have two cases:

	$1 < |B_{(e)}| < |B|$ for some $e \in A-B $:
	By the lemma and starting hypothesis, $|A_{(e)}+B_{(e)}| = |A_{(e)}| + |B_{(e)}| - 1 $ and by the inductive hypothesis $B_{(e)} $ and $A_{(e)} $ are arithmetic progressions with the same step.
	Hence $A+B = A_{(e)} + B_{(e)}$ is an arithmetic progression, reducing us back into the previous case.

	$|B_{(e)}| = |B|$ or 1 $\forall e \in A-B$.
	Let $E \subseteq A - B$ be the set of $e $ s.t. $|B_{(e)}| = |B| $.
	Then $B + E \subseteq A $ and by Cauchy-Davenport, we have that $|B| + |E| - 1 \le |A| \iff |E| \le |A| - |B| + 1 $.
	Since $|A - B| \ge |A| + |B| - 1 $ by Cauchy-Davenport, by pidgeonhole principle there are at least $2|B| - 2$ values of $e $.
	By pidgeonhole again, we have $e,e' $ s.t. $B_{(e)} = B_{(e')} = \{b\} $.

	Since $|A+B| = |A| + |B| - 1 $, $A+B = A_{(e)} + b = A_{(e')} + b $ and thus $A\cup (B+e) = A \cup (B+e') $.
	As $|B_{(e)}| = 1 $, $A \cap B+e = b + e$ and similarly for $e' $, $B+e $ and $B+e' $ differ by at most one element (use the fact that $A\cup (B+e) = A \cup (B+e')$).
	Hence $B $ is an arithmetic sequence of $e' - e $.
\end{proof}
\end{document}
